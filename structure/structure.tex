\documentclass[]{beamer}

\usepackage[english]{babel}
\usepackage[utf8]{inputenc}
\usepackage{csquotes}
\usepackage[T1]{fontenc}
\usepackage{lmodern}
\usepackage{default}
\usepackage{graphicx}
\usepackage[backend=bibtex,style=ieee]{biblatex}
\usepackage[acronym,shortcuts,nowarn]{glossaries}
\usepackage{enumerate}

\graphicspath{{figures/}{imgs/}}
\bibliography{bib/db.bib}
\makeglossaries
\InputIfFileExists{abbrev/acronyms}{}{}

\usetheme{INSO}

\title{Presentation of the Structure}
%\subtitle{Short title}
\author{Jakob Gruber}
\matrnr{0203440}
\date{\today}

\setlength{\parskip}{1em}

\begin{document}

\maketitle

\begin{frame}[allowframebreaks]{Paper Structure}
    State of the Art in Dining Cryptographer Networks
    \begin{itemize}
    \item Introduction

        \begin{itemize}
        \item Why is anonymity important: Communicate and organize without fear of reprisal.
        \item Why dc-nets: Mix-nets are vulnerable to traffic analysis.
        \item Generally: What are dc-nets, and what has been achieved in research?
        \item Structure of remaining paper.
        \end{itemize}

    \item The Dining Cryptographers Protocol
    
        \begin{itemize}
        \item Short intuitive example as in original paper.
        \item Definitions of basic terms: Anonymity set, Key-sharing Graph, \ldots
        \item Formalized description of dc-protocol.
        \end{itemize}
    
    \item Recent Developments
    
        \begin{itemize}
        \item 1989: Unreliable channel 1990: Detection of disruptors.
        \item 2003: k-anonymity. Weakened security goals for efficiency.
        \item 2004: Proactive accountability in Theory.
        \item 2010: Dissent: Scalability \& accountability with strong guarantees in practise.
        \item 2013: Verdict: Proactive accountability and improved scalability in practise.
        \end{itemize}

    \item Related Work

        \begin{itemize}
        \item Mix-Networks: Scalable, but susceptible to traffic analysis.
        \item Crowds: Again, traffic analysis.
        \item Herbivore: Anonymity only within small subgroups.
        \end{itemize}

    \item Conclusion

    \item References
    \end{itemize}
\end{frame}

\begin{frame}[allowframebreaks]{Slide Structure}
    Dining Cryptographer Networks
    
    \begin{itemize}
    \item Introduction (1-2 slides)
    \item Dining Cryptographer Protocol (4-5 slides)
    \item k-Anonymous Message Transmission (2-3 slides)
    \item Dissent (4-5 slides)
    \item Verdict (2-3 slides)
    \item Related Work (2-3 slides)
    \item Conclusion
    \end{itemize}
    
\end{frame}

\begin{titleframe}
    \begin{center}
    \alert{\Large Thank you!}
    \end{center}
\end{titleframe}

\end{document}
