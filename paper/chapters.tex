\section{Introduction}

Anonymous communications are a vital tool in protecting privacy, the rights of minorities,
freedom of speech, and the organization of resistance against repressive regimes.

In general, the goal of anonymous group communications is the transmission of a message
in a group of $n$ members such that an adversary gains no knowledge of where the message originated.
In many cases, adversaries have access to, or even control of the basic communication
links between participants.

Unfortunately, \acp{MixNet} \cite{journals/cacm/Chaum81} and
widely used derivatives such as \Tor \cite{conf/uss/DingledineMS04} are vulnerable to traffic
analysis attacks in which adversaries monitor network links in order to gain information
about communication participants \cite{murdoch2005low}. \acp{DCNetwork} on the other hand
are able to guarantee anonymity even in the face of strong traffic analysis.

Of course, providing such strong guarantees also comes with a cost, and \acp{DCNetwork}
have their own set of inherent challenges. In order to defeat traffic analysis,
each participant must transmit an identical amount of data during each round,
leading to substantial scalability issues. Likewise, the \ac{DCNetwork}'s strong anonymity guarantees result
in difficulties in detecting and ejecting internal attackers (also called disruptors) from the network.

Research in \acp{DCNetwork} has mostly focused on scalability and handling of disruptors,
and in this paper we present a selection of approaches towards their improvement.
In Section \ref{sec:dining_cryptographers_protocol}, we give an introduction
to the \ac{DCProtocol} and its main properties. Section \ref{sec:k-anonymous_messaging} 
covers k-anonymous messaging, a \ac{DCProtocol} variant which obtains better scalability
through relaxed anonymity guarantees. Finally, in Section \ref{sec:dissent} we discuss a practical
\ac{DCProtocol} implementation called \Dissent, offering a retroactive blame mechanism.

\section{The Dining Cryptographers Protocol} \label{sec:dining_cryptographers_protocol}

% Adversary, model, assumptions, guarantees, goals...
% static and public key sharing graph, reliable network, honest participants?

\subsection{Protocol}

The initial idea of \acp{DCNetwork} was first proposed
at Crypto 1984 and subsequently published in \citeyear{journals/joc/Chaum88} \cite{journals/joc/Chaum88}.
\citeauthor{journals/joc/Chaum88} introduced the protocol with the following demonstration:
Consider three cryptographers seated around a table at a restaurant.
After finishing dinner, they are informed that their check has already been paid\footnote{
We assume that the check has been paid exactly once.}.
Baffled, they wish to determine whether the payer was one of them (in which case
they do not want to reveal his identity) or if their dinner has been sponsored by the NSA.

They use the following protocol to answer this question.
In the first stage, each of them establishes a shared secret with each of his neighbors by flipping
a coin behind a menu such that only the two of them can see its outcome. 
Each cryptographer
who has not paid then simply announces whether the two coin flips he participated
in landed on the same side or not, while the paying cryptographer says exactly the opposite.
They then count the number of announced differences; if it is odd, then one of them has paid.
Otherwise, they determine that their dinner was paid by the NSA.

We now formalize this scenario through a keysharing graph: let the cryptographers be represented as the vertices 
$v_1, v_2, v_3$ in the complete graph $\mathbb{K}_3$. Each edge $e_{ij} = (v_i, v_j)$ represents a shared
key and is assigned a bit value $k(e_{ij}) = 0$ for ``heads'' and $k(e_{ij}) = 1$ for ``tails''. Without loss of generality,
let $v_1$ be the transmitting vertex, and let its message be $m_1 \in \{0, 1\}$ while $m_2, m_3 = 0$.
Every vertex $v_i$ now creates its announcement value $a(v_i)$ by combining all associated shared keys $k(e_{ij})$
as well as its message $m_i$ with XOR operations\footnote{
The sum modulo two which is sometimes used in the literature is equivalent to XOR.}
and broadcasts $a(v_i)$ to all other vertices in the network. Each $v_i$ is now able to
reconstruct the originally transmitted message by calculating
$m'_i = m_1 = a(v_1) \oplus \cdots \oplus a(v_3)$. Since $k(e_{ij}) = k(e_{ji})$ and $m_2, m_3 = 0$:

\begin{align*}
m'_i &= a(v_1) \oplus a(v_2) \oplus a(v_3) \\
     &= m_1 \oplus m_2 \oplus m_3
     \oplus (k(e_{12}) \oplus k(e_{21})) \\
     & \oplus (k(e_{13}) \oplus k(e_{31})) 
     \oplus (k(e_{23}) \oplus k(e_{32})) = m_1
\end{align*}

The \ac{DCProtocol} is unconditionally secure if each member is honest and performs the
protocol correctly. A short proof sketch considers a nonpaying cryptographer $v_1$
wishing to expose the person who paid. In the first case both of $v_1$'s shared keys are the same
(i.e. $a(v_1) = 0$) while $a(v_2) = 1$ and $a(v_3) = 0$. If $k(e_{23}) = k(e_{12}) = k(e_{13})$, then $v_2$
is the payer, with $v_3$ paying otherwise. Since $k(e_{23})$ is determined uniformly at random,
$v_1$ does not gain any information. In the second case, the two keys held by $v_1$ are different.
Without loss of generality, let $k(e_{12}) = 1$ and $k(e_{13}) = 0$.
If $a(v_2) = a(v_3) = 1$, the payer is $v_2$ if $k(e_{23}) = 1$ and $v_3$ otherwise.
Likewise, if $a(v_2) = a(v_3) = 0$ then the payer is $v_2$ if $k(e_{23}) = 0$ and $v_3$ otherwise.
Again, $v_1$ does not gain any information about which person has paid the bill.
A more general proof is available in \cite{journals/joc/Chaum88}.

The presented protocol is easily generalized to a larger number of participants $n > 3$ by simply
using the complete graph $\mathbb{K}_n$. Multi-bit message $M$ of length $|M| = j$
may be transmitted by running $j$ iterations of the \ac{DCProtocol}.  


\subsection{Discussion}

The \ac{DCProtocol} exhibits some very desirable properties. For instance, it is one of the few
protocols able to guarantee anonymity in the absence of a trusted party \cite{juels2004dining}.

It provides unconditional security when using unconditional secrecy channels\footnote{
If public key encryption is used, the \ac{DCProtocol} remains computationally secure.},
i.e. it remains secure even in the presence of a
strong attacker who can eavesdrop on all communications and has unlimited computational power.

Since all participants transmit an identical volume of data in each round the \ac{DCProtocol}
is completely immune to traffic analysis, unlike other anonymous messaging protocols such
as \acp{MixNet} \cite{journals/cacm/Chaum81}.

Malicious pooling of secret keys between two or more participants in order to
gain information about the message transmitter is called collusion.
Consider the anonymity set\footnote{Given a set of keys and their corresponding edges $K$ in a keysharing graph $G = (V, E)$, the anonymity set is the set of vertices in a connected component of $(V, E \setminus K)$ \cite{journals/joc/Chaum88}.}
relative to the keysharing graph and the set of pooled key edges.
The attacker gains only the knowledge which anonymity set the transmitter belongs to \cite{journals/joc/Chaum88}. To completely expose the transmitter, the attacker would
have to own every single key shared by the transmitter.

It is possible to alter the key-sharing graph to accomodate varying requirements to
security and efficiency as long as the graph stays connected. Sparser graphs require fewer
shared keys, but make exposure through collusion less difficult. In the extreme case of
a ring topology in which each participant only shares keys with two other members, it is
possible to expose a sender by collusion of his two neighbors.

Besides sender-anonymity, it is also possible to achieve recipient-anonymity by
encrypting the message with the intended recipient's public key and with each
participant attemptying to decrypt the message upon arrival \cite{journals/joc/Chaum88}.
Similarly, it is possible to tie a message to a pseudonym by using public key cryptography
to sign it with the private key of a the desired pseudonym.

However, important questions remain as well. So far, we have simply assumed secret shared keys
to exist, but realistically these have to be generated and shared securely. Solutions
again mostly revolve around public key cryptography and will not be discussed further in
this paper.

Likewise, we have assumed the group membership and structure (i.e. the key-sharing graph $G$)
to remain static and known to all participants and ignored any issues of joining or leaving participants.

Another challenge is the prevention of collisions between message transmissions.
In this basic protocol, we have simply assumed that only a single participant transmits
in each round. Obviously, in practice some way of collision avoidance or recovery must exist.
Chaum suggests a simple retransmission scheme on detected collisions \cite{journals/joc/Chaum88}
while \Dissent uses a shuffle protocol to agree on
a fixed transmission schedule \cite{journals/corr/abs-1004-3057}.

Performance and scalability are major issues as well. A \ac{DCNetwork} of $n$ participants requires $\Omega(n^3)$ messages per round, which obviously does not scale well to larger groups.
Much of the research on \acp{DCNetwork} has focused on this area; k-anonymous messaging \cite{von2003k},
\Dissent \cite{journals/corr/abs-1004-3057} and \Verdict \cite{corrigan2013proactively}
all attempt to make higher group memberships feasible.

We have already briefly mentioned the effect of accidental transmission collisions; what about the
case in which these are triggered on purpose by a malicious attacker intent on disrupting the
protocol? It is very difficult for \acp{DCNetwork} to deal with these \ac{DoS} attacks,
since one of their main goals is the guarantee of sender anonymity. \citeauthor{journals/joc/Chaum88}
proposed an elaborate system of traps and checks in order to detect and ultimately exclude
disruptors from the protocol \cite{journals/joc/Chaum88}. \Dissent and \Verdict use the concept of
accountability in order to recover from these attacks more efficiently \cite{journals/corr/abs-1004-3057,corrigan2013proactively}.

\section{k-anonymous Messaging} \label{sec:k-anonymous_messaging}

% Definitions: k-anonymity, honest/faulty participant (dissent)
% Equivalence member, node, participant

k-anonymous message transmission is a modification of the \ac{DCProtocol} which was published in \citeyear{von2003k} by \citeauthor{von2003k}. Its main contributions are better scalability properties through relaxation
of the core anonymity goals \cite{von2003k}.

Unlike \acp{DCNetwork} which guarantee that an attacker monitoring all traffic cannot gain
any information which of the $n$ participants was the sender, \citeauthor{von2003k} introduce
the concept of sender and receiver k-anonymity. A protocol is said to be sender (receiver) k-anonymous
if each message sent (received) by an honest party cannot be distinguished from those
sent (received) by at least $k - 1$
other honest parties \cite{von2003k}. While full anonymity is required in some cases, in others
k-anonymity suffices even for small values of $k$. For example, 2-anonymity and 3-anonymity would invalidate, respectively, a criminal and a civil charge in the United States legal system.

The attack model consists of an adversary with polynomially bounded computational power
who can monitor all communications and
has control over up to $\frac{n}{2}$ participants\footnote{
These faulty participants must be chosen once at the beginning of the protocol and may not
be altered during its execution, i.e. the adversary is non-adaptive.}. The adversary may not affect
the delivery of messages, and communications between participants are always delivered.
A trusted public-key infrastructure is assumed to exist\footnote{Public-key cryptography is
required for the special case of handling non-participation by faulty nodes which is not discussed further
in this context.}.

\subsection{Protocol}

k-anonymous messaging relies on a combination of a secure multiparty sum algorithm
together with zero-knowledge proofs and Pedersen commitments.

A secure multiparty sum protocol allows a group of participants to collaboratively
compute the sum of their respective inputs without learning anything about the
inputs of other group members. Consider participants $P_1, \ldots, P_n$, each with
inputs $X_i$ taken from the additive group of integers modulo $m$ $\mathbb{Z}_m$.
Each participant splits his input into $n$ values $s_{i,1}, \ldots, s_{i,n}$ such that $\sum_j s_{i,j} = X_i$
and then sends element $s_{i,j}$ to participant $P_j$. Every $P_j$ then forms
$s'_j = \sum_i s_{i,j}$, broadcasts $s'_j$, and computes the final sum $X = \sum_j s'_j = \sum_i X_i$.

Zero-knowledge proofs allow a prover to convince a verifier of some statement while
giving the verifier no additional knowledge. \citeauthor{chaum1987demonstrating}
present a protocol that demonstrates knowledge of the discrete logarithm of $x = h^r \mod p$
where $q$ divides $p - 1$ and $p,q$ are prime in \cite{chaum1987demonstrating}.

The Pedersen commitment \cite{pedersen1992non} allows party $P$ to publicly commit
to a value $s$ by publishing $C_r(s)$ such that a) $P$ is bound to his commitment, i.e. he cannot later
claim an $s' \neq s$ to be the original value, and b) the commitment does not reveal
any information about $s$. This commitment scheme is based on the difficulty of computing discrete
logarithms.

In k-anonymous messaging, the group consisting of $n$ members is first partitioned
into subgroups of size $M = O(k)$ such that with high probability $k$ members of each group
are honest. This is accomplished by combining the original assumption that at least
$\frac{n}{2}$ members are honest with a Chernoff Bound to limit the probability that
any subgroup contains less than $k$ honest members.

Each subgroup then runs $2M$ iterations of the protocol with each acting as a separate transmission slot.
Every participant $i$ randomly chooses one personal transmission slot $t_i \in [1, 2M]$,
and can provide a zero-knowledge
proof that at most one slot was chosen. The input of each member $i$ in round $\phi$ is
a tuple $X_{i\phi} = (m_{i\phi}, g_{i\phi})$ interpreted as an integer,
where $m_{i\phi}$ is the message to be transmitted, and
$g_{i\phi}$ identifies the group of the intended recipient. Each member $i$ sets $X_{it_i}$ to $(m_i, g_i)$,
and $X_{i\phi} = (nil, nil)$ for all other rounds $\phi \neq t_i$.

In each round $\phi$, the subgroup then performs the multiparty sum protocol to compute
$\sum_i X_{i\phi}$. Pedersen commitments are used to verify a) partitions of individual sum
elements $X_{i\phi}$, b) partial sums, and c) the final result. If no collision takes place,
$\sum_i X_{i\phi} = (m_i, g_i)$ for the transmitting member $i$ in round $\phi$, since all
other inputs $X_{i\phi} = (nil, nil)$. Each member of the subgroup finally broadcasts the
reconstructed message to each member of the target group. 

In case of a collision, the transmitting member simply
attempts retransmission in a following round. If all participants are honest,
each member $i$ has a probability of at least $1/2$ of successfully transmitting $m_i$
in $t_i$ since there are $2M$ rounds and $M$ members.

\subsection{Discussion}

Recall that in a \ac{DCNetwork}, the count of messages sent is $\Omega(n^3)$ in any case.
In a k-anonymous messaging round with exclusively honest participants, $O(k^2)$ messages are transmitted.
Faulty members are detected with high probability, and as such
honest rounds are treated as the standard case in efficiency analysis.
Note that $k$ is unrelated to $n$, vastly improving the protocol's scalability.
On the other hand, it is possible to increase $k$ up to $n$ if needed.

Unlike the base \ac{DCNetwork}, adversaries are restricted to polynomially bound computational
power in order to facilitate the use of public-key cryptography, Pedersen commitments, and zero-knowledge
proofs. Likewise, the number of faulty participants is limited to a constant fraction in order
to bound the probability of any subgroup containing less than $k$ honest members.

In contrast to the base \ac{DCProtocol}, k-anonymous messaging offers improved scalability and
a novel way of detecting faulty members. As a trade-off, anonymity guarantees are relaxed
and provide k-anonymity only. It is especially relevant for cases in which group sizes are high,
and complete anonymity is not required. \citeauthor{von2003k} highlight use over the
Internet as particularly efficient because of the low round complexity in a k-anonymity protocol.

\section{Dissent} \label{sec:dissent}

\Dissent is a fairly recent research contribution which provides an implementation of a modified \ac{DCNetwork},
offering ``integrity, anonymity, and accountability in the face of strong traffic analysis and compromised members'' \cite{journals/corr/abs-1004-3057}. In this context, integrity means that each message either
arrives at every honest group member, or all members know that the protocol round has failed. Anonymity
requires that a group of $l \leq n - 2$ colluding attackers cannot match a message to its sender
with better accuracy than random guessing. Finally, the protocol is accountable if no honest member is
ever wrongfully accused, and in the case of disruption every honest member exposes at least one faulty member.

% TODO Drop these definitions to save space?

Again, the attack model assumes a very strong attacker with polynomially bounded computational
power able to eavesdrop on all traffic and corrupt any subset of protocol participants.

Group membership, round management and public key systems are outside
the scope of this topic and as such several simplifying assumptions are made: group membership is closed
and known to all, members
know when to start a round and can distinguish between rounds, all members participate in all rounds
and do not go silent, and all members have public encryption as well as signing keys.

\Dissent uses two subprotocols: the first, called the shuffle protocol, is used to assign
transmission slot reservations to each group member. The second, called the bulk protocol,
facilitates the subsequent bulk transmission of all messages.

\subsection{Shuffle Protocol}

The shuffle protocol is based on \cite{brickell2006efficient} but refines this approach
with a go/no go phase in order to counteract \ac{DoS} attacks by faulty members.

The goal of this protocol is to allow each participant to create a transmission slot reservation
such that each participant is assigned exactly one slot while not learning anything about
the assignments of any other participant.

Initially, each participant has a primary encryption key pair $(x_i, y_i)$, a signing key pair
$(u_i, v_i)$, and a message $m_i$. The length of all messages is fixed to $L$.

All members must have agreed on a session nonce $n_R$, a common ordering of all members
$1 \ldots n$, and every participant's public encryption and signing keys. \citeauthor{journals/corr/abs-1004-3057} suggest achieving this agreement via techniques
proposed in \cite{lamport1998part,castro1999practical}.

Every participant maintains a tamper-evident log as in \cite{haeberlen2007peerreview}
containing each message sent and received. During each phase $\phi$ of the shuffle protocol,
each honest member $i$ sends at most one message $\mu_{i\phi}$.
Every $\mu_{i\phi}$ contains a hash $h_{i\phi}$ of the current log head in addition to other data
and is signed with $i$'s signing key. The protocol proceeds in $5$ ($ + 1$) phases:

\emph{Secondary Key Generation.}
      Each member $i$ generates a secondary encryption key pair $(w_i, z_i)$ and broadcasts
      the public key $z_i$.
      
\emph{Data Submission.}
      Each member $i$ encrypts his message $m_i$ in $n$ layers with all secondary public encryption keys,
      starting with $z_n$ and ending with $z_1$, resulting in $C'_i$.  $C'_i$ is then further encrypted
      with all primary public keys ($y_n, \ldots, y_1$), resulting in $C_i$ which is subsequently
      sent to member $1$. 
      
\emph{Anonymization.}
      This phase consists of $n$ subphases. In subphase $i$, member $i$ receives all elements,
      places them in a vector $\vec{C_i}$, randomly permutes its elements, strips one layer of
      encryption, and sends the resulting elements to member $i + 1$. The final vector $\vec{C_n}$
      is broadcast to the entire group.
      
\emph{Verification.}
      $\vec{C_n}$ should now consist of a permutation of all $C'_i$. Each member $i$ verifies that
      his $C'_i$ is present, and sets the $GO_i$ flag accordingly. $GO_i$ is then broadcast
      together with a hash over all broadcast messages received or sent during the previous phases.
      If all members receive the go ahead together with the expected broadcast message log hash
      from all other members, we proceed to the final stage.
      
\emph{Decryption.}
      Each member $i$ broadcasts his secondary private key $w_i$, and finally decrypts the remaining
      layers of encryption on $\vec{C_n}$ to receive a permutation of $m_i$.
      
\emph{Blame.}
      If a failure occurs at any point during this protocol, participants use their stored
      log to replay actions by all other players and expose faulty members.

\subsection{Bulk Protocol}

This phase relies on both the shuffle protocol and a modification of the \ac{DCProtocol}
to transmit all messages $m_i$ to the group while preserving integrity, anonymity and accountability.

Contrary to the shuffle protocol, messages do not have a fixed length. Instead, each member
$i$ has a message $m_i$ together with its length $L_i$. Every member also holds encryption and
signing key pairs as well as knowing the session nonce and a common member ordering.

The protocol proceeds in phases:

\emph{Message Descriptor Generation.}
      In this phase, each member $i$ generates a shared key $C_{ij}$ for each other member $j \neq i$
      by choosing a seed value $s_{ij}$ and using a pseudorandom number generator \cite{stinson2006cryptography} to generate a
      key sequence of length $L_i$. The ciphertext $C_{ii}$ is then generated by
      XORing the message $m_i$ together with all shared keys $C_{ij}$. Finally, the message descriptor
      $d_i$ is formed by combining $L_i$ and $\text{HASH}(m_i)$ with the vector $\vec{H_i}$ of all secret key
      hashes as well as the vector $\vec{S_i}$ containing all seeds $s_{ij}$ encrypted with the respective
      public key $y_j$.

\emph{Message Descriptor Shuffle.}
      The shuffle protocol is now executed with each participant using its message descriptor
      as the input. Upon conclusion, each member is assigned a transmission slot $\pi(i)$.

\emph{Data transmission.}
      Every participant $j$ can now decode $s_{ij}$ and use it to generate the shared key $C_{ij}$ which
      is subsequently verified using $\text{HASH}(C_{ij})$. All $C_{ij}$ are finally signed and
      broadcast in the order of $\pi$.
      
\emph{Message Recovery.}
      All messages may now be recovered by simply XORing all received messages from the last round for each phase: $m_i = C_{i1} \oplus \ldots \oplus C_{in}$. Received ciphertexts $C_{ij}$ can be verified
      through their hashes taken from the $\vec{H_i}$ vector in the message descriptor.

\emph{Blame.}
      In case any failure occurs during previous phases, the protocol is aborted and the blame phase is entered.
      Members whose message was corrupted broadcast an accusation of member $j$ together with the
      associated seed $s_{ij}$ through the shuffle protocol.
      All other participants can then replay the encryption process to check whether $j$
      is faulty or not.

\subsection{Discussion}
% TODO: Maybe expand this section?
The total complexity of a protocol run is $O(n^2) + \sum_i L_i$ in the normal case
and $O(n^3) + \sum_i L_i$ with faulty members. Unlike k-anonymous messaging,
\Dissent provides provable accountability, faulty members are always exposed, and full anonymity
within the group is guaranteed.

A working prototype implementation demonstrates the feasibility of this protocol in practice
for small to moderate group sizes, scaling from a runtime of 1 minute in a 4-member group
to 14 minutes in a 40-member group \cite{journals/corr/abs-1004-3057}.

\section{Related Work}

\Verdict is one of the most recent developments in the field of \acp{DCNetwork}.
It extends the mechanisms used in \Dissent to provide proactive accountability, i.e.
that every participant can verify at any time that messages by other participants are well-formed \cite{corrigan2013proactively}. Due to the performance overhead of proactively accountable messages,
\Verdict offers a hybrid mode in which the standard XOR-based \ac{DCProtocol}
is employed by default, only switching to the verifiable primitive when under attack.
In comparison to \Dissent, the main advantages \Verdict offers are 
improved scalability and efficiency properties when under attack. The time taken to
identify faulty participants in a 1000-node group is reduced from 20 minutes to 26 seconds.

Some early developments in the field of \acp{DCNetwork} were left unmentioned in this paper:
\cite{waidner1989dining} exposes and presents solutions to several weaknesses in the
original protocol as well as removing the assumption
of a reliable network. In \cite{bos1990detection}, \citeauthor{bos1990detection} present a system of creating sender slot reservations.

Herbivore \cite{goel2003herbivore} provides anonymous communications through a variant of the
\ac{DCProtocol} with small, dynamic groups. Honest participants attempt to avoid jamming by
leaving disrupted groups. However, powerful attackers can exploit this to disrupt exactly those groups
which are only partially corrupted, making participants likely to settle in completely corrupted subgroups.
Like k-anonymous networks, Herbivore scales well due to its low group sizes.

\acp{MixNet} are another form of anonymity networks, initially described by \citeauthor{journals/cacm/Chaum81}
\cite{journals/cacm/Chaum81}. Messages by participants are shuffled through a series of mix servers
in order to hide the original sender. In contrast to \acp{DCNetwork}, security in \acp{MixNet} requires
at least a constant fraction of nodes to be honest. \acp{MixNet} are susceptible to traffic analysis attacks,
in which attackers are able gain information about a system by monitoring its traffic.
\Tor \cite{conf/uss/DingledineMS04} is a \ac{MixNet} variant and one of the most well-known anonymity networks
in use today.

\enlargethispage{-.9in}

\section{Conclusion}

In this paper, we have given a brief summary of \acp{DCNetwork} and important
research results in the field.

\acp{DCNetwork}'s advantages are their potential for strong anonymity guarantees
and their immunity to traffic analysis attacks as well as their adaptibility in 
adjusting to different security requirements.

Advances have been made in improving their scalability, and \acp{DCProtocol} 
are now suitable for moderate sized groups with tolerance for high
transmission latencies. If full $n$-anonymity is not needed, relaxed
security requirements makes the usage of \acp{DCNetwork} feasible even with
a large number of participants.

Disruptor detection may now be handled more efficiently through cryptographic techniques
such as zero-knowledge proofs, commitment schemes and the concept of accountability.
Practical applicability
has been shown through prototype implementions of both \Verdict and \Dissent, offering
acceptable performance for groups of up to 1000 members.

\begin{comment}
Important points:

* Non-interactive
* Computationally/Unconditionally secure
* No central trusted party
* Shuffled send
* Security goals: integrity, anonymity, accountability (See dissent 2.3)
* Attack model (dissent 2.3)
* Assumptions, highlight difference between them (Faulty nodes never silent, ..., dissent 2.4)

Sections:

* 88: Base protocol
* 89: Disco (unconditional untraceability, computationally secure serviceability)
* 89: Waidner (Unreliable channel)
* 90: Detection of disruptors
* (03: Herbivore)
* 03: k-anonymity (maybe this is related work instead? weakens security goals to gain efficiency). Small DC subnets.
* 04: dc-revisited (proactive accountability. efficient cheater detection + recovery)
* 10: Dissent
* 13: Verdict

Definitions:

* Anonymity set
* Anonymity terms (cite def paper)
* Anonymity game (dissent [7])
* k-anonymity (k-anonymity)
* Robustness? (k-anonymity 3.3)
* (Partial, Full) Collusion
* Disruption
* Anytrust assumption
* Zero knowledge proofs

Attacks:

* Sybil (dissent [17])
* Sock puppetry (dissent [36])
* Traffic analysis (verdict [4, 34, 38])

Related work:

* Mix nets
* Crowds (dissent [29])
* CliqueNet (k-anon [17])
* Verifiable shuffles
* Group signatures, ring signatures
* Herbivore

Random notes:

* k-anonymous provide anonymity only when most members are honest (see dissent, related work)
* Herbivore provides anonymity only within small subgroups

\end{comment}
